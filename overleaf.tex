%%%%%%%%%%%%%%%%%%%%%%%%%%%%%%%%%%%%%%%%%%%%%%%%%%%%%%%%%%%%%%%%%%%%%%
% Overleaf Example: A quick guide to LaTeX
%
% Original Source: Dave Richeson (divisbyzero.com), Dickinson College
% Modified By: Paul Gessler, Overleaf (overleaf.com)
%
% A one-size-fits-all LaTeX cheat sheet. Kept to two pages, so it
% can be printed (double-sided) on one piece of paper
%
% Feel free to distribute this example, but please keep the referral
% to divisbyzero.com
%
% Guidance on the use of the Overleaf logos can be found here:
% https://www.overleaf.com/for/partners/logos
%%%%%%%%%%%%%%%%%%%%%%%%%%%%%%%%%%%%%%%%%%%%%%%%%%%%%%%%%%%%%%%%%%%%%%
% How to use Overleaf:
%
% You edit the source code here on the left, and the preview on the
% right shows you the result within a few seconds.
%
% Bookmark this page and share the URL with your co-authors. They can
% edit at the same time!
%
% You can upload figures, bibliographies, custom classes and
% styles using the files menu.
%
% If you're new to LaTeX, the wikibook is a great place to start:
% http://en.wikibooks.org/wiki/LaTeX
%
%%%%%%%%%%%%%%%%%%%%%%%%%%%%%%%%%%%%%%%%%%%%%%%%%%%%%%%%%%%%%%%%%%%%%%

\documentclass[10pt,landscape,letterpaper]{article}
\usepackage{amssymb,amsmath,amsthm,amsfonts}
\usepackage{multicol,multirow}
\usepackage{spverbatim}
\usepackage{graphicx}
\usepackage{ifthen}
\usepackage[landscape]{geometry}
\usepackage[colorlinks=true,urlcolor=olgreen]{hyperref}
\usepackage{booktabs}
\usepackage{fontspec}
\setmainfont[Ligatures=TeX]{TeX Gyre Pagella}
\setsansfont{Fira Sans}
\setmonofont{Inconsolata}
\usepackage{unicode-math}
\setmathfont{TeX Gyre Pagella Math}
\usepackage{microtype}
% dirty fix for microtype
\makeatletter
\def\MT@is@composite#1#2\relax{%
  \ifx\\#2\\\else
    \expandafter\def\expandafter\MT@char\expandafter{\csname\expandafter
                    \string\csname\MT@encoding\endcsname
                    \MT@detokenize@n{#1}-\MT@detokenize@n{#2}\endcsname}%
    % 3 lines added:
    \ifx\UnicodeEncodingName\@undefined\else
      \expandafter\expandafter\expandafter\MT@is@uni@comp\MT@char\iffontchar\else\fi\relax
    \fi
    \expandafter\expandafter\expandafter\MT@is@letter\MT@char\relax\relax
    \ifnum\MT@char@ < \z@
      \ifMT@xunicode
        \edef\MT@char{\MT@exp@two@c\MT@strip@prefix\meaning\MT@char>\relax}%
          \expandafter\MT@exp@two@c\expandafter\MT@is@charx\expandafter
            \MT@char\MT@charxstring\relax\relax\relax\relax\relax
      \fi
    \fi
  \fi
}
% new:
\def\MT@is@uni@comp#1\iffontchar#2\else#3\fi\relax{%
  \ifx\\#2\\\else\edef\MT@char{\iffontchar#2\fi}\fi
}
\makeatother

\ifthenelse{\lengthtest { \paperwidth = 11in}}
    { \geometry{margin=0.4in} }
	{\ifthenelse{ \lengthtest{ \paperwidth = 297mm}}
		{\geometry{top=1cm,left=1cm,right=1cm,bottom=1cm} }
		{\geometry{top=1cm,left=1cm,right=1cm,bottom=1cm} }
	}
\pagestyle{empty}
\makeatletter
\renewcommand{\section}{\@startsection{section}{1}{0mm}%
                                {-1ex plus -.5ex minus -.2ex}%
                                {0.5ex plus .2ex}%x
                                {\sffamily\large}}
\renewcommand{\subsection}{\@startsection{subsection}{2}{0mm}%
                                {-1explus -.5ex minus -.2ex}%
                                {0.5ex plus .2ex}%
                                {\sffamily\normalsize\itshape}}
\renewcommand{\subsubsection}{\@startsection{subsubsection}{3}{0mm}%
                                {-1ex plus -.5ex minus -.2ex}%
                                {1ex plus .2ex}%
                                {\normalfont\small\itshape}}
\makeatother
\setcounter{secnumdepth}{0}
\setlength{\parindent}{0pt}
\setlength{\parskip}{0pt plus 0.5ex}
% -----------------------------------------------------------------------

\usepackage{academicons}

\usepackage{draftwatermark,afterpage,xcolor}
\definecolor{olgreen}{HTML}{4f9c45}
\SetWatermarkText{\aiOverleaf}
\SetWatermarkFontSize{0.9\paperheight}
\SetWatermarkAngle{0}
\SetWatermarkColor[HTML]{EDF5EC}
%\SetWatermarkLightness{0.95}
\SetWatermarkHorCenter{0.35\paperwidth}

\begin{document}
\footnotesize
%\raggedright

\begin{center}
  {\huge\sffamily\bfseries A quick guide to} \huge\bfseries \LaTeX \\
\end{center}
\setlength{\premulticols}{0pt}
\setlength{\postmulticols}{0pt}
\setlength{\multicolsep}{1pt}
\setlength{\columnsep}{1.8em}
\begin{multicols}{3}

\section{What is {\rmfamily \LaTeX}?}

\LaTeX\ (usually pronounced ``LAY teck,'' sometimes ``LAH teck,'' and never ``LAY tex'') is a \emph{format}, or collection of macro commands, for \TeX, the standard for most professional mathematics and scientific writing.
\TeX\ is a powerful typesetting engine created by Donald Knuth of Stanford University (his first version appeared in 1978).
Leslie Lamport was responsible for creating \LaTeX, a popular set of user commands for \TeX. A team of \LaTeX\ programmers created the current version, \LaTeX\ 2$\varepsilon$.

\section{Mathematics}

\subsection{Math vs. text vs. functions}
In properly typeset mathematics, the variables appear in italics (for example, $f(x)=x^{2}+2x-3$).
The exception to this rule is predefined functions (for example, $\sin (x)$).
Thus it is important to \emph{always} treat text, variables, and functions correctly.
See the difference between $x$ and x, -1 and $-1$, and $sin(x)$ and $\sin(x)$.

There are two ways to present a mathematical expression -- \emph{inline} or as a \emph{display}.

\subsection{Inline mathematical expressions}

Inline math expressions occur as part of the normal flow of text.
To produce an inline expression, place the math expression between dollar signs (\verb!$!).
For example, typing \spverb!$90^{\circ}$ is the same as $\frac{\pi}{2}$ radians!  yields $90^{\circ}$ is the same as $\frac{\pi}{2}$ radians.

\subsection{Displayed mathematical expressions}

Displays are mathematical expressions that are given their own line and are centered on the page.
These are usually used for important equations that deserve to be showcased on their own line or for large equations that cannot fit inline.
To produce displayed mathematics, place the mathematical expression  between the symbols  \verb!\[! and \verb!\]!.
Typing \verb!\[x=\frac{-b\pm\sqrt{b^2-4ac}}{2a}\]! yields \[x=\frac{-b\pm\sqrt{b^2-4ac}}{2a}.\]

\subsection{Display style}
To get full-size inline math, use \verb!\displaystyle!.
Use this sparingly.
Typing \spverb!this $\displaystyle \sum_{n=1}^{\infty}\frac{1}{n}$, and not this $\sum_{n=1}^{\infty}\frac{1}{n}$! yields\\
this $\displaystyle \sum_{n=1}^{\infty}\frac{1}{n}$, and not this $\sum_{n=1}^{\infty}\frac{1}{n}$.

\section{Images}

You can put images (pdf, png, jpg, or gif) in your document.
They need to be in the same location as your .tex file when you compile the document.
Omit \verb![width=.5in]! if you want the image to be full-sized.

\verb!\begin{figure}[tbp]!\\
\verb!\includegraphics[width=.5in]{imagename.jpg}!\\
\verb!\caption{The (optional) caption goes here.}!\\
\verb!\end{figure}!

\section{Text decorations}

Your text can be \textit{italic} (\verb!\textit{italic}!), \textbf{bold} (\verb!\textbf{bold}!), or \underline{underlined} (\verb!\underline{underlined}!).

Your math can contain bold, $\mathbf{R}$ (\verb!\mathbf{R}!), or blackboard bold, $\mathbb{R}$ (\verb!\mathbb{R}!).
You may want to used these to express the sets of real numbers ($\mathbb{R}$ or $\mathbf{R}$), integers ($\mathbb{Z}$ or $\mathbf{Z}$), rational numbers ($\mathbb{Q}$ or $\mathbf{Q}$), and natural numbers ($\mathbb{N}$ or $\mathbf{N}$).

For text appearing inside a math expression, use \verb!\text!.\\
\verb!(0,1]=\{x\in\mathbb{R}:x>0\text{ and }x\le 1\}! yields\\
$(0,1]=\{x\in\mathbb{R}:x>0\text{ and }x\le 1\}$.

(Without the \verb!\text! command it treats ``and'' as three variables: $(0,1]=\{x\in\mathbb{R}:x>0 and x\le 1\}$.)

\section{Spaces and new lines}

\LaTeX\ ignores extra spaces and new lines. For example,

\verb!This   sentence will       look!

\verb!fine after      it is     compiled.!

This   sentence will       look
fine after      it is     compiled.

Leave one full empty line between two paragraphs. Place \verb!\\! at the end of a line to create a new line (but not create a new paragraph).

\verb!This!

\verb!compiles!

~

\verb!like\\!

\verb!this.!

This
compiles

like\\
this.

Use  \verb!\noindent! to prevent a paragraph from indenting.

\section{Comments}

Use \verb!%! to create a comment. Nothing on the line after the \verb!%! will be typeset. \verb!$f(x)=\sin(x)$ %this is the sine function! yields $f(x)=\sin(x)$%this is the sine function
.

\section{Delimiters}

\begin{tabular}{lll}
\toprule
\emph{description} & \emph{command} & \emph{output}\\
\midrule
parentheses &\verb!(x)! & (x)\\
square brackets &\verb![x]! & [x]\\
curly braces& \verb!\{x\}! & \{x\}\\
\bottomrule\addlinespace
\end{tabular}

To automatically make delimiters large enough to fit the content, use them together with \verb!\right! and \verb!\left!. For example, \spverb!\left\{ \sin \left( \frac{1}{n} \right) \right\} _{n}^{\infty}! produces \[ \left\{\sin\left(\frac{1}{n}\right)\right\}_{n}^{\infty}.\]

Curly braces are non-printing characters that are used to gather text that has more than one character. Observe the differences between the four expressions \verb!x^2!, \verb!x^{2}!, \verb!x^2t!, \verb!x^{2t}! when typeset: $x^2$, $x^{2}$, $x^2t$, $x^{2t}$.

\section{Lists}

You can produce ordered and unordered lists.

\begin{tabular}{lll}
\toprule
\emph{description} & \emph{command} & \emph{output}\\
\midrule
unordered list&
\begin{tabular}{@{}l}
\verb!\begin{itemize}!\\
\verb!  \item Thing 1!\\
\verb!  \item Thing 2!\\
\verb!\end{itemize}!
\end{tabular}&
\parbox{0.75in}{\begin{itemize}
  \item Thing 1
  \item Thing 2
\end{itemize}}
~\\
ordered list&
\begin{tabular}{@{}l}
\verb!\begin{enumerate}!\\
\verb!  \item Thing 1!\\
\verb!  \item Thing 2!\\
\verb!\end{enumerate}!
\end{tabular}&
\parbox{0.75in}{\begin{enumerate}
  \item Thing 1
  \item Thing 2
\end{enumerate}} \\
\bottomrule
\end{tabular}

\section{Symbols (in \emph{math} mode)}

\subsection{The basics}
\begin{tabular}{lll}
\toprule
\emph{description} & \emph{command} & \emph{output}\\
\midrule
addition & \verb!+! & $+$\\
subtraction & \verb!-! & $-$\\
plus or minus & \verb!\pm! & $\pm$\\
multiplication (times) & \verb!\times! & $\times$\\
multiplication (dot) & \verb!\cdot! & $\cdot$\\
division symbol & \verb!\div! & $\div$\\
division (slash) & \verb!/! & $/$\\
circle plus & \verb!\oplus! & $\oplus$\\
circle times & \verb!\otimes! & $\otimes$\\
equal & \verb!=! & $=$\\
not equal & \verb!\ne! & $\ne$\\
less than & \verb!<! & $<$\\
greater than & \verb!>! & $>$\\
less than or equal to & \verb!\le! & $\le$\\
greater than or equal to & \verb!\ge! & $\ge$\\
approximately equal to & \verb!\approx! & $\approx$\\
infinity & \verb!\infty! & $\infty$\\
dots & \verb!1,2,3,\ldots! & $1,2,3,\ldots$\\
dots & \verb!1+2+3+\cdots! & $1+2+3+\cdots$\\
fraction & \verb!\frac{a}{b}! & $\frac{a}{b}$\\
square root & \verb!\sqrt{x}! & $\sqrt{x}$\\
$n$th root & \verb!\sqrt[n]{x}! & $\sqrt[n]{x}$\\
exponentiation & \verb!a^b! & $a^{b}$\\
subscript & \verb!a_b! & $a_{b}$\\
absolute value & \verb!|x|! & $|x|$\\
natural log  & \verb!\ln(x)! & $\ln(x)$\\
logarithms & \verb!\log_{a}b! & $\log_{a}b$\\
exponential function & \verb!e^x=\exp(x)! & $e^{x}=\exp(x)$\\
degree & \verb!\deg(f)! & $\deg(f)$\\
\bottomrule
\end{tabular}
\vfill
{\hfill\includegraphics[width=0.9\hsize]{Overleaf_main_logo_vector}\hfill}
\newpage

\subsection{Functions}
\begin{tabular}{lll}
\toprule
\emph{description} & \emph{command} & \emph{output}\\
\midrule
maps to & \verb!\to! & $\to$\\
composition& \verb!\circ! & $\circ$\\
\multirow{4}{0.5in}{piecewise function} & \verb!|x|=\begin{cases}! & \multirow{4}{*}{$\displaystyle |x|=\begin{cases}x&x\ge 0\\-x&x<0\end{cases}$}\\
&\verb!x & x\ge 0\\!&\\
&\verb!-x & x<0!&\\
&\verb!\end{cases}!&\\
\bottomrule
\end{tabular}

\subsection{Greek and Hebrew letters}
\begin{tabular}{llll}
\toprule
\emph{command} & \emph{output}&\emph{command} & \emph{output}\\
\midrule
\verb!\alpha! & $\alpha$&\verb!\tau! & $\tau$\\
\verb!\beta! & $\beta$&\verb!\theta! & $\theta$\\
\verb!\chi! & $\chi$&\verb!\upsilon! & $\upsilon$\\
\verb!\delta! & $\delta$&\verb!\xi! & $\xi$\\
\verb!\epsilon! & $\epsilon$&\verb!\zeta! & $\zeta$\\
\verb!\varepsilon! & $\varepsilon$&\verb!\Delta! & $\Delta$\\
\verb!\eta! & $\eta$&\verb!\Gamma! & $\Gamma$\\
\verb!\gamma! & $\gamma$&\verb!\Lambda! & $\Lambda$\\
\verb!\iota! & $\iota$&\verb!\Omega! & $\Omega$\\
\verb!\kappa! & $\kappa$&\verb!\Phi! & $\Phi$\\
\verb!\lambda! & $\lambda$&\verb!\Pi! & $\Pi$\\
\verb!\mu! & $\mu$&\verb!\Psi! & $\Psi$\\
\verb!\nu! & $\nu$&\verb!\Sigma! & $\Sigma$\\
\verb!\omega! & $\omega$&\verb!\Theta! & $\Theta$\\
\verb!\phi! & $\phi$&\verb!\Upsilon! & $\Upsilon$\\
\verb!\varphi! & $\varphi$&\verb!\Xi! & $\Xi$\\
\verb!\pi! & $\pi$&\verb!\aleph! & $\aleph$\\
\verb!\psi! & $\psi$&\verb!\beth! & $\beth$\\
\verb!\rho! & $\rho$&\verb!\daleth! & $\daleth$\\
\verb!\sigma! & $\sigma$&\verb!\gimel! & $\gimel$ \\
\bottomrule
\end{tabular}

\subsection{Set theory}
\begin{tabular}{lll}
\toprule
\emph{description} & \emph{command} & \emph{output}\\
\midrule
set brackets & \verb!\{1,2,3\}! & $\{1,2,3\}$\\
element of & \verb!\in! & $\in$\\
not an element of & \verb!\not\in! & $\not\in$\\
subset of & \verb!\subset! & $\subset$\\
subset of & \verb!\subseteq! & $\subseteq$\\
not a subset of & \verb!\not\subset! & $\not\subset$\\
contains & \verb!\supset! & $\supset$\\
contains & \verb!\supseteq! & $\supseteq$\\
union & \verb!\cup! & $\cup$\\
intersection & \verb!\cap! & $\cap$\\
big union &
\verb!\bigcup_{n=1}^{10}A_n! &
$ \bigcup_{n=1}^{10}A_{n}$\\
\addlinespace
big intersection & \verb!\bigcap_{n=1}^{10}A_n! &$ \bigcap_{n=1}^{10}A_{n}$\\
empty set & \verb!\emptyset! & $\emptyset$\\
power set & \verb!\mathcal{P}! & $\mathcal{P}$\\
minimum & \verb!\min! & $\min$\\
maximum & \verb!\max! & $\max$\\
supremum & \verb!\sup! & $\sup$\\
infimum & \verb!\inf! & $\inf$\\
limit superior & \verb!\limsup! & $\limsup$\\
limit inferior & \verb!\liminf! & $\liminf$\\
closure & \verb!\overline{A}! & $\overline{A}$\\
\bottomrule
\end{tabular}

\subsection{Calculus}
\begin{tabular}{lll}
\toprule
\emph{description} & \emph{command} & \emph{output}\\
\midrule
derivative & \verb!\frac{df}{dx}! & $\displaystyle \frac{df}{dx}$\\
\addlinespace
derivative & \verb!f'! & $f'$\\
\addlinespace
partial derivative &
%\begin{tabular}{l}
\verb!\frac{\partial f}{\partial x}!
%\end{tabular}
& $\displaystyle \frac{\partial f}{\partial x}$\\
\addlinespace
integral & \verb!\int! & $\int$\\
double integral & \verb!\iint! & $\iint$\\
triple integral & \verb!\iiint! & $\iiint$\\
limits & \verb!\lim_{x\to \infty}! & $\displaystyle \lim_{x\to \infty}$\\
summation  &
\verb!\sum_{n=1}^{\infty}a_n! &
$\displaystyle \sum_{n=1}^{\infty}a_n$\\
product  &
\verb!\prod_{n=1}^{\infty}a_n! &
$\displaystyle \prod_{n=1}^{\infty}a_n$\\
\bottomrule
\end{tabular}

\subsection{Logic}
\begin{tabular}{lll}
\toprule
\emph{description} & \emph{command} & \emph{output}\\
\midrule
not & \verb!\sim! & $\sim$\\
and & \verb!\land! & $\land$\\
or & \verb!\lor! & $\lor$\\
if...then & \verb!\to! & $\to$\\
if and only if & \verb!\leftrightarrow! & $\leftrightarrow$\\
logical equivalence & \verb!\equiv! & $\equiv$\\
therefore & \verb!\therefore! & $\therefore$\\
there exists  & \verb!\exists! & $\exists$\\
for all & \verb!\forall! & $\forall$\\
implies & \verb!\Rightarrow! & $\Rightarrow$\\
equivalent & \verb!\Leftrightarrow! & $\Leftrightarrow$\\
\bottomrule
\end{tabular}

\subsection{Linear algebra}
\begin{tabular}{lll}
\toprule
\emph{description} & \emph{command} & \emph{output}\\
\midrule
vector & \verb!\vec{v}! & $\vec{v}$\\
vector & \verb!\mathbf{v}! & $\mathbf{v}$\\
norm & \verb!||\vec{v}||! & $||\vec{v}||$\\
matrix&
\begin{tabular}{@{}l}
\verb!\left[!\\
\verb!\begin{array}{ccc}!\\
\verb!1 & 2 & 3 \\!\\
\verb!4 & 5 & 6\\!\\
\verb!7 & 8 & 0!\\
\verb!\end{array}!\\
\verb!\right]!\end{tabular}&
$\displaystyle \left[\begin{array}{ccc}1 & 2 & 3 \\4 & 5 & 6 \\7 & 8 & 0\end{array}\right]$\\
\\determinant&
\begin{tabular}{@{}l}
\verb!\left|!\\
\verb!\begin{array}{ccc}!\\
\verb!1 & 2 & 3 \\!\\
\verb!4 & 5 & 6 \\!\\
\verb!7 & 8 & 0!\\
\verb!\end{array}!\\
\verb!\right|!
\end{tabular}&
$\displaystyle \left|\begin{array}{ccc}1 & 2 & 3 \\4 & 5 & 6 \\7 & 8 & 0\end{array}\right|$\\
determinant & \verb!\det(A)! & $ \det(A)$\\
trace & \verb!\operatorname{tr}(A)! & $\operatorname{tr}(A)$\\
dimension & \verb!\dim(V)! & $\dim(V)$\\
\bottomrule
\end{tabular}

\subsection{Number theory}
\begin{tabular}{lll}
\toprule
\emph{description} & \emph{command} & \emph{output}\\
\midrule
divides & \verb!\mid! & $\mid$\\
does not divide & \verb!\not \mid! & $\not \mid$\\
div & \verb!\operatorname{div}! & $\operatorname{div}$\\
mod & \verb!\mod! & $\operatorname{mod}$\\
greatest common divisor & \verb!\gcd! & $\gcd$\\
ceiling & \verb!\lceil x \rceil! & $\lceil x\rceil$\\
floor & \verb!\lfloor x \rfloor! & $\lfloor x \rfloor$\\
\bottomrule
\end{tabular}

\subsection{Geometry and trigonometry}
\begin{tabular}{lll}
\toprule
\emph{description} & \emph{command} & \emph{output}\\
\midrule
angle& \verb!\angle ABC! & $\angle ABC$\\
degree& \verb!90^{\circ}! & $90^{\circ}$\\
triangle& \verb!\triangle ABC! & $\triangle ABC$\\
segment& \verb!\overline{AB}! & $\overline{AB}$\\
sine& \verb!\sin! & $\sin$\\
cosine& \verb!\cos! & $\cos$\\
tangent& \verb!\tan! & $\tan$\\
cotangent& \verb!\cot! & $\cot$\\
secant& \verb!\sec! & $\sec$\\
cosecant& \verb!\csc! & $\csc$\\
inverse sine& \verb!\arcsin! & $\arcsin$\\
inverse cosine& \verb!\arccos! & $\arccos$\\
inverse tangent& \verb!\arctan! & $\arctan$\\
\bottomrule
\end{tabular}

\section{Symbols (in \emph{text} mode)}

These symbols do \emph{not} have to be surrounded by dollar signs.

\begin{tabular}{lll}
\toprule
\emph{description} & \emph{command} & \emph{output}\\
\midrule
dollar sign & \verb!\$! & \$ \\
percent & \verb!\%! & \% \\
ampersand & \verb!\&! & \& \\
pound & \verb!\#! & \# \\
backslash & \verb!\textbackslash! & \textbackslash \\
left quote marks & \verb!``! & `` \\
right quote marks & \verb!''! & '' \\
single left quote  & \verb!`! & ` \\
single right quote  & \verb!'! & ' \\
hyphen & \verb!X-ray! & X-ray\\
en-dash & \verb!pp. 5--15! & pp. 5--15 \\
em-dash & \verb!Yes---or no?! & Yes---or no? \\
\bottomrule
\end{tabular}

\section{Getting started with \texorpdfstring{\raisebox{-0.18ex}{\protect\includegraphics[height=2.1ex]{Overleaf_main_logo_vector}}}{Overleaf} and {\rmfamily \LaTeX}}
\LaTeX\ collaborative authoring online: \url{https://overleaf.com}\\
Overleaf \LaTeX\ documentation: \url{https://overleaf.com/learn}\\
Learn \LaTeX\ in 30 minutes:\\ \url{https://overleaf.com/learn/latex/Learn_LaTeX_in_30_minutes}

%Great symbol look-up site: \href{http://detexify.kirelabs.org/}{Detexify}\\
%\href{http://amath.colorado.edu/documentation/LaTeX/Symbols.pdf}{\LaTeX\ Mathematical Symbols}\\
%\href{ftp://tug.ctan.org/pub/tex-archive/info/symbols/comprehensive/symbols-letter.pdf}{The Comprehensive \LaTeX\ Symbol List}\\
%\href{http://mirrors.med.harvard.edu/ctan/info/lshort/english/lshort.pdf}{The Not So Short Introduction to \LaTeX\ 2$\varepsilon$}\\
\vfill
{Comprehensive \TeX\ Archive Network}: \url{http://www.ctan.org/}\\
{\TeX\ Users Group: }\url{http://www.tug.org/}

\vfill
Want to work offline? Local install options for Linux or Windows: \href{https://www.tug.org/texlive}{\TeX\ Live};
MacOS: \href{http://www.tug.org/mactex/}{Mac\TeX};
Windows: \href{http://miktex.org/}{MiK\TeX}
\vfill
\hrule
\vfill
Based on the version by Dave Richeson, Dickinson College, \mbox{\url{http://divisbyzero.com/}}.
Reproduced with permission.
\end{multicols}

%\SetWatermarkHorCenter{0.65\paperwidth}
\end{document}
